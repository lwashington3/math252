%! Author = Len Washington III
%! Date = 9/22/2023

% Preamble
\documentclass[12pt]{report}

% Packages
\usepackage[title={Sep 22, 2023 Notes}]{math252notes}

% Document
\begin{document}

\setcounter{chapter}{3}
\chapter[Higher Order DEs]{Higher Order Differential Equations}\label{ch:higher-order-differential-equations}
%<*Section-4.4>
\setcounter{section}{3}
\section{Nonhomogeneous, Linear DE with Constant Coefficients}\label{sec:nonhomogeneous-linear-de-with-constant-coefficients}

\subsection{Method of Undetermined Coefficients}\label{subsec:method-of-undetermined-coefficients}
Section 4.5 gives another approach but it is a bit more abstract

\hyperref[linear-equations]{\[ a_{n}(x)\frac{d^{n}y}{dx^{n}} + a_{n-1}(x)\frac{d^{n-1}y}{dx^{n-1}} + a_{n-2}(x)\frac{d^{n-2}y}{dx^{n-2}} + \dots + a_{1}(x)\frac{dy}{dx}+ a_{0}y = g(x)\mbox{ where } g(x) \neq 0\]}

\theorem{If we can find any one particular solution $y_{p}$ of this DE $(y_{p} + y_{c})$, where $y_{c}$ is the solution of the complementary DE\label{dfn:complementary-de} (the same LHS$=0$ instead of $g(x)$), is also a solution of the non-homogeneous DE, then the general solution is \begin{equation*}
\begin{aligned}
	y &= y_{c} + y_{p}\\
	  &= c_{1}y_{1} + c_{2}y_{2} + c_{3}y_{3} + \dots + c_{n}y_{n} + y_{p}
\end{aligned}
\end{equation*}} where you use \hyperref[sec:higher-order-linear-homogeneous-de-with-constant-coefficients]{Section 4.3} methods for the $c_{i}y_{i}$'s.

\example
\[ y'' + 4y' - 2y = 2x^{2} -3x + 6\]
\subsubsection{Step 1: Find the General Solution $y_{c}$ of the complimentary DE} $y'' + 4y' - 2y = 0$

Aux equation: \begin{equation*}
\begin{aligned}
	m^{2} + 4m - 2 &= 0\\
	m^{2} + 4m + 4 &= 6\\
	(m+2)^{2} &= 6\\
	m+2 &= \pm\sqrt{6}\\
	m &= -2\pm\sqrt{6}\\
	y_{1} &= e^{(-2 + \sqrt{6})x}\\
	y_{2} &= e^{(-2 - \sqrt{6})x}\\
\end{aligned}
\end{equation*}

\subsubsection{Step 2: Find a particular solution $y_{p}$ of given DE}
Educated Guess: \[ y_{p} = Ax^{2} + Bx + C \] for some coefficients $A$, $B$, $C$. For the moment, they're undetermined coefficients.\\

Plugging in the $y_{p}$, we get \begin{equation*}
\begin{aligned}
	y_{p}' &= 2Ax + B\\
	y_{p}'' &= 2A\\
\end{aligned}
\end{equation*}
So,
\begin{equation*}
\begin{aligned}
	2A + 4(2Ax + B) - 2\left( Ax^{2} + Bx + C \right) &= 2x^{2} - 3x + 6\\
	2A + 8Ax + 4B - 2Ax^{2} - 2Bx - 2C &= 2x^{2} - 3x + 6\\
	-2Ax^{2} + 8Ax - 2Bx + 2A + 4B - 2C &= 2x^{2} - 3x + 6\\
	-2Ax^{2} + (8A - 2B)x + (2A + 4B - 2C) &= 2x^{2} - 3x + 6\\
	-2A &= 2\\
	8A - 2B &= -3\\
	2A + 4B - 2C &= 6\\
	-2A &= 2\\
	A &= -1\\
	8(-1) - 2B &= -3\\
	-8 - 2B &= -3\\
	8 + 2B &= 3\\
	2B &= -5\\
	B &= -\frac{5}{2}\\
	2(-1) + 4(-\frac{5}{2}) - 2C &= 6\\
	-2 + -10 - 2C &= 6\\
	-2C &= 18\\
	C &= -9\\
\end{aligned}
\end{equation*}

\subsubsection{Step 3: Check}
\begin{equation*}
\begin{aligned}
	y_{p}' &= 2(-1)x + \left( -\frac{5}{2} \right)\\
		   &= -2x - \frac{5}{2}\\
	y_{p}'' &= 2(-1)\\
		   &= -2\\
	y'' + 4y' - 2y &= -2 + 4\left( -2x - \frac{5}{2} \right) - 2\left( -x^{2} - \frac{5}{2}x - 9 \right)\\
				   &= -2 - 8x - 10 + 2x^{2} + 5x + 18\\
				   &= 2x^{2} - 8x + 5x - 10 + 18 -2\\
				   &= 2x^{2} - 3x + 6\\
\end{aligned}
\end{equation*}

\example
\[ y'' - y' + y = 2\sin(3x) \]
\subsubsection{Step 1: Find the General Solution $y_{c}$ of the complimentary DE}
Aux equation: \begin{equation*}
\begin{aligned}
	m^{2} - m + 1 &= 0\\
	m &= \frac{1 \pm \sqrt{(-1)^{2} - 4(1)(1)}}{2(1)}\\
	  &= \frac{1 \pm \sqrt{1 - 4}}{2}\\
	  &= \frac{1 \pm \sqrt{-3}}{2}\\
	  &= \frac{1 \pm \sqrt{3}i}{2}\\
	m_{1} &= \frac{1 + \sqrt{3}i}{2}\\
	m_{2} &= \frac{1 - \sqrt{3}i}{2}\\
	y_{1} &= e^{\frac{1}{2}x}\cos\left( \frac{\sqrt{3}}{2}x \right)\\
	y_{2} &= e^{\frac{1}{2}x}\sin\left( \frac{\sqrt{3}}{2}x \right)\\
\end{aligned}
\end{equation*}

\subsubsection{Step 2: Guess $y_{p}=A\sin(3x) + B\cos(3x)$}
Plug into the DE \begin{equation*}
\begin{aligned}
	&y''& &-y'& &+y &=2\sin(3x) \\
	&\overbrace{-9A\sin(3x)-9B\cos(3x)}& - &\left( \overbrace{3A\cos(3x) - 3B\sin(3x)} \right)& + & \overbrace{A\sin(3x) + B\cos(3x)} &= 2\sin(3x)\\
\end{aligned}
\end{equation*}

%</Section-4.4>

\end{document}