%! Author = Len Washington III
%! Date = 9/01/2023

% Preamble
\documentclass[title={Sep 01{,} 2023 Notes}]{math252notes}

% Document
\begin{document}

\setcounter{chapter}{1}
\chapter{First-Order Differential Equations}
\setcounter{section}{3}
%<*Section-2.4>
\section{Exact Equations}\label{sec:exact-equations}
1st Order D.E. in differential form \[ M(x,y)dx+N(x,y)dy=0 \] Given a function \[ z=f(x,y) \], the total differential, $dz$, is defined as \[ dz = \frac{\partial f}{\partial x}dx + \frac{\partial f}{\partial y}dy \]

\subsection{Method}\label{subsec:exact-equations-method}
See if we can find a function $f(x,y)$ such that \[ \frac{\partial f}{\partial x}=M, \frac{\partial f}{\partial y}=N \]

If we can do this, then the D.E. is equivalent to \[ df=0 \Rightarrow f(x,y)=c \] is an implicit solution of D.E.\\

Assume that $M$ and $N$ have continuous 1st order partials (assuming $f$ exists) \begin{equation*}
\begin{aligned}
	\left. \begin{array}{rlc}
		My&=\frac{\partial}{\partial y}\frac{\partial f}{\partial x}dy=&f_{xy}\\
		Nx&=\frac{\partial}{\partial x}\frac{\partial f}{\partial y}dy=&f_{yx}\\
	\end{array} \right\} \mbox{ Theorem tells us these are equal}
\end{aligned}
\end{equation*}
This provides a quick test to check if the D.E. is exact or not.

\example
\begin{equation*}
\begin{aligned}
	2xydx+\left( x^{2}-1 \right)dy&=0\\
	M(x,y)=2xy \sep N(x,y)= x^{2}-1\\
\end{aligned}
\end{equation*}
To check if the D.E. is exact
\[ M_{y}=2x= N_{x} \]
We now know there exists a function $f(x,y)$ with \begin{equation*}
\begin{aligned}
	 \frac{\partial f}{\partial x}&=M=&2xy\\
	 \frac{\partial f}{\partial y}&=N=&x^{2}-1
\end{aligned}
\end{equation*}
\begin{equation*}
\begin{aligned}
	f_{M}(x,y)&=\int \frac{\partial f}{\partial x}dx\\
			  &=\int 2xydx\\
			  &=x^{2}y+\phi(y)\\
	\frac{\partial f}{\partial y}\left( x^{2}y+\phi(y) \right) &= x^{2}-1 \mbox{ required to equal }N\\
	x^{2}+\phi'(y)&=x^{2}-1\\
	\phi'(y)&=-1\\
	\phi(y)&=\int-1dy\\
		   &=-y\\
	f(x,y)&=x^{2}y-y\\
	d\left( f(x,y)\right)&=0\\
	f(x,y)&=c\\
	x^{2}y-y&=c\mbox{ is an implicit solution of the D.E.}\\
\end{aligned}
\end{equation*} \textbf{Note: the $f_{M}$ format is just there to show which partial equation was integrated. It was made by me and, as far as I know, is not standardly known.}

\example
\begin{equation*}
\begin{aligned}
	\left( e^{2y}-y\cos(xy) \right)dx + (2xe^{2y}-x\cos(xy)+2y)dy&=0\\
	M_{y}&=N_{x}\\
	\frac{\partial}{\partial y}\left( e^{2y}-y\cos(xy) \right) = \frac{\partial}{\partial x}\left( 2xe^{2y}-x\cos(xy)+2y \right)\\
	2e^{2y}- \left[ \cos(xy) - y\sin(xy)\times{x} \right]  = 2e^{2y}- \left( \cos(xy) - x\sin(xy)\times{y} \right) + 0\\
	2e^{2y} -\cos(xy) + xy\sin(xy) = 2e^{2y} -\cos(xy) + xy\sin(xy)\\
\end{aligned}
\end{equation*}\begin{equation*}
\begin{aligned}
	\frac{\partial f}{\partial x}=M&=e^{2y}-y\cos(xy)\\
	\frac{\partial f}{\partial y}=N&=2xe^{2y}-x\cos(xy)+2y\\
	f_{N}(x,y) &= \int \frac{\partial f}{\partial y}dy\\
			   &= \int \left(2xe^{2y} -x\cos(xy) + 2y\right) dy\\
			   &= \frac{2xe^{2y}}{2} - \frac{x\sin(xy)}{x} + 2\times\frac{y^{2}}{2} + \phi(x)\\
			   &= xe^{2y} - \sin(xy) + y^{2} + \phi(x)\\
\end{aligned}
\end{equation*}
Take the $\partial x$ of this and equate with $M$:\begin{equation*}
\begin{aligned}
	M&=\frac{\partial}{\partial x}\left( xe^{2y} - \sin(xy) + y^{2} + \phi(x) \right)\\
	e^{2y}-y\cos(xy) &= e^{2y} - y\cos(xy) + 0 + \phi'(x) \\
	0 &= \phi'(x) \\
	\phi(x) &= c \\
\end{aligned}
\end{equation*}
So $f(x,y)=c_{2}$ is the solution \[ xe^{2y}-\sin(xy)+y^{2}=c \]\\
\[ dx = \frac{\partial z}{\partial x}dx + \frac{\partial z}{\partial y}dy \]

\subsection{What can you do if $M_{y} \neq N_{x}$}\label{subsec:what-can-you-do-if-exact-solution-dont-match}
\textbf{Sometimes} you can multiply the DE by an integrating factor $\mu(x,y)$ to get an exact DE.\\\\

\noindent If \[ \frac{M_{y}-N_{x}}{N} \] is a function of only $x$, then \[ \mu = e^{\int \frac{M_{y}-N_{x}}{N} dx}\] will be an I.F.\\

\noindent If \[ \frac{N_{x}-M_{y}}{M} \] is a function of only $y$, then \[ \mu = e^{\int \frac{N_{x}-M_{y}}{M} dy}\] will be an I.F.

\example
\[ xydx + \left( 2x^{2}+3y^{2}-20 \right)dy=0 \]

\begin{equation*}
\begin{aligned}
	M_{y}&=x\\
	N_{x}&=4x\\
	M_{y}&\neq N_{x}\\
	\frac{N_{x}-M_{y}}{M}&=\frac{4x-x}{xy}\\
						 &=\frac{3x}{xy}\\
						 &=\frac{3}{y}\mbox{ is a function of just }y\\
\end{aligned}
\end{equation*}

So: \begin{equation*}
\begin{aligned}
	\mu&=e^{\int \frac{3}{y}dy}\\
	   &=e^{3\ln y}\\
	   &=y^{3}\\
	xy^{4}dx + y^{3}\left( 2x^{2}+3y^{2}-20 \right)dy&=0(y^{3})\\
	xy^{4}dx + \left( 2x^{2}y^{3}+3y^{5}-20y^{3} \right)dy&=\\
	M_{y}&=N_{x}\\
	4xy^{3} &= 4xy^{3}\\
	\frac{\partial f}{\partial x} % TODO: Add function from finish
\end{aligned}
\end{equation*}
%</Section-2.4>

\end{document}