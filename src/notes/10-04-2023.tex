%! Author = Len Washington III
%! Date = 10/04/2023

% Preamble
\documentclass[12pt]{report}

% Packages
\usepackage[title={Oct 04, 2023 Notes}]{math252notes}

% Document
\begin{document}

\setcounter{chapter}{5}
\chapter{Series Solutions of Linear Equations}\label{ch:series-solutions-of-linear-equations}
%<*Section-6.1>
\section{Solution by Infinite Series}\label{sec:solution-by-infinite-series}
2nd order linear DE with (possibly) variable coefficients
\[ a_{2}(x)y'' + a_{1}(x)y' + a_{0}(x)y = f(x) \]
\[ y'' + P(x)y' + Q(x)y = F(x) \]

\subsection{Review of Infinite Series Facts}\label{subsec:review-of-infinite-series-facts}
\subsubsection{Maclaurin Series}
\[ \sum_{n=0}^{\infty} a_{n}x^{n}  \]
Power series centered at 0

\subsubsection{Taylor Series}
\[ \sum_{n=0}^{\infty} a_{n}(x-a)^{n}  \]
Centered at a = 0

It's a theorem that power series either
\begin{enumerate}[label=(\arabic*)]
    \item Converge all real numbers $x$ on the interval $I=(-\infty, \infty)$ and the radius of convergence is $R=\infty$
	\item Converge only when $x=a$ on the interval $I=[a,a]$ and the radius of convergence is $R=0=\{a\}$
	\item The series converges on an interval centered at a finite, non-zero radius $R=(a-R, a+R)$
\end{enumerate}

\subsection{Ratio Test}\label{subsec:ratio-test}
Use the Ratio Test to determine which of these 3 cases occurs in a specific problem.

The 3 cases of the ratio test are:
\begin{itemize}
	\item[$L < 1$], the series converges
	\item[$L > 1$], the series diverges
	\item[$L = 1$],
\end{itemize}

\example
Determine the radius and interval of convergence for \[ \sum_{n=0}^{\infty} \frac{x^{n}}{3^{n}(n+1)}  \]

\begin{equation*}
\begin{aligned}
	L &= \lim_{n\rightarrow\infty} \frac{\bigg| \frac{x^{n+1}}{3^{n+1}(n+2)} \bigg|}{\bigg| \frac{x^{n}}{3^{n}(n+1)} \bigg|}\\
	  &= \lim_{n\rightarrow\infty} \frac{|x|}{3}\frac{n+1}{n+2}\\
	  &= \frac{|x|}{3}\lim_{n\rightarrow\infty}\frac{n+1}{n+2}\\
	  &= \frac{|x|}{3}(1)\\
	  &= \frac{|x|}{3}\\
\end{aligned}
\end{equation*}\begin{equation*}
\begin{aligned}
	\frac{|x|}{3} &< 1\\
	|x| &< 3\\
	I &= \left( -3, 3 \right)\\
\end{aligned}
\end{equation*}

	%</Section-6.1>

\end{document}