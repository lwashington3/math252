%! Author = Len Washington III
%! Date = 9/20/2023

% Preamble
\documentclass[12pt]{report}

% Packages
\usepackage{titling}
\title{Sep 20, 2023 Notes}
\usepackage{math252notes}

% Document
\begin{document}

\setcounter{chapter}{3}
\chapter[Higher Order DEs]{Higher Order Differential Equations}\label{ch:higher-order-differential-equations}
%<*Section-4.3>
\setcounter{section}{2}
\section[Higher Order DEs with Constant Coefficients]{Higher Order, Linear, Homogeneous DE with Constant Coefficients}\label{sec:higher-order-linear-homogeneous-de-with-constant-coefficients}
\example
\[ 3y^{(4)} - 2y''' + 7y' + 8y = 0 \]

Theorems in \hyperref[sec:linear-equations]{4.1} tell us that the general solution is of the form $y=c_{1}y_{1}$.

\conjecture{A solution of the form $y=e^{mx} \Rightarrow y'=me^{mx}$. }
\example
\begin{equation*}
\begin{aligned}
	5y' - 4y &= 0\\
	y' - \frac{4}{5}y &= 0\\
	me^{mx} - \frac{4}{5}e^{mx} &= 0\\
	e^{mx} \left( m - \frac{4}{5} \right) &= 0\\
	m - \frac{4}{5} &= 0\\
	m &= \frac{4}{5}\\
\end{aligned}
\end{equation*}
$y=c_{1}e^{\frac{4}{5}x}$ is the general solution of the DE

\example
\begin{equation*}
\begin{aligned}
	y'' + 5y' - 6y &= 0\\
	y(m^{2}e^{mx}) + 5(me^{mx}) - 6e^{mx} &= 0\\
	e^{mx} \left( m^{2}y + 5m - 6 \right) &= 0\\
	m^{2}y + 5m - 6 &= 0\\
	(m+6)(m-1) &= 0\\
	m+6=0 \sep m-1= 0\\
	m=-6 \sep m=1\\
	y_{1}=e^{-6x} \sep y_{2}e^{x}\mbox{ These are \hyperref[dfn:linear-independent]{Linearly Independent (L.I)}}\\
\end{aligned}
\end{equation*}
Therefore: \[ y=c_{1}e^{-6x} + c_{2}e^{x} \]

\example
\begin{equation*}
\begin{aligned}
	y'' - 6y' + 9y &= 0\\
	m^{2}e^{mx} - 6\left( me^{mx} \right) + 9e^{mx} &= 0\\
	m^{2} - 6m + 9 &= 0\\
	(m-3)^{2} &= 0 \mbox{ } m=3 \mbox{ is a repeated root}\\
	m-3 &= 0\\
	m &= 3\\
	y_{1} = e^{3x} \sep y_{2} = e^{3x} \mbox{ are linearly dependent}\\
	\mbox{Use the }&\mbox{\hyperref[eq:reduction-of-order]{Reduction of order function}:}
\end{aligned}
\end{equation*}
\begin{equation*}
\begin{aligned}
	y_{2} &= y_{1}\int \frac{e^{-\int P(x)dx}}{(y_{1}(x))^{2}} dx \\
		  &= e^{3x}\int \frac{e^{-\int -6dx}}{(e^{3x})^{2}} dx \\
		  &= e^{3x}\int \frac{e^{\int 6dx}}{e^{6x}} dx \\
		  &= e^{3x}\int \frac{e^{6x}}{e^{6x}} dx \\
		  &= e^{3x}\int 1 dx \\
		  &= e^{3x}x \\
		  &= xe^{3x} \\
\end{aligned}
\end{equation*}
Always works out for this solution\label{dfn:higher-order-lienar-homog-const-coef} if $e^{m_{1}x}$ is a solution and $m_{1}$ is a root of multiplicity $k$ than $y_{1} = e^{m_{1}x},y_{2} = xe^{m_{1}x},\dots,y_{k}=x^{k-1}e^{mx}$ are linear solutions.

\begin{equation*}
\begin{aligned}
	y'' + 9y &= 0\\
	m^{2} + 9 &= 0\\
	m^{2} &= -9\\
	m &= \sqrt{-9} \mbox{ No real solutions}\\
	m &= \pm \sqrt{-9}\\
	m &= \pm 3i\\
	y &= c_{1}e^{3ix} + c_{2}e^{-3ix} \mbox{ where } c_{1} \& c_{2} \mbox{ arbitrary complex numbers }
\end{aligned}
\end{equation*}
We'd rather only deal with real-valued solutions.

\subsection{Euler's Formula}\label{subsec:euler's-formula}
\[ e^{i\theta} = \cos(\theta) + i\sin(\theta) \]

\begin{equation*}
\begin{aligned}
	e^{i3x} &= \cos(3x) + i\sin(3x)\\
	e^{-i3x} &= \cos(-3x) + i\sin(-3x)\\
	e^{-i3x} &= \cos(3x) - i\sin(3x)\\
	e^{i3x} + e^{-i3x} &= \cos(3x) + i\sin(3x) + \cos(3x) - i\sin(3x)\\
	e^{i3x} + e^{-i3x} &= 2\cos(3x)\\
	Y_{1} = \frac{1}{2}e^{i3x} + \frac{1}{2}e^{-i3x} &= \cos(3x)\\
	Y_{2} &= \sin(3x)\\ % TODO: Find out where y_{2} initially came from
	\frac{1}{2i}y_{1} - \frac{1}{2i}y_{2} &= \sin(3x)
\end{aligned}
\end{equation*}
General solution: \begin{equation*}
\begin{aligned}
	y &= C_{1}Y_{1} + C_{2}Y_{2}\\
	  &= C_{1}\cos(3x) + C_{2}\sin(3x)\\
\end{aligned}
\end{equation*} where $C_{1}$ and $C_{2}$ are complex numbers that generate all complex-valued solutions of the DE

\example
\begin{equation*}
\begin{aligned}
	y'' + 25y &= 0\\
	m^{2}e^{mx} + 25e^{mx} &= 0\\
	m^{2} + 25 &= 0\\
	m^{2} &= -25\\
	m &= \pm 5i\\
	\mbox{General }&\mbox{solution}\\
	y_{1} &= c_{1}y_{1} + c_{2}y_{2}\\
		  &= c_{1}\cos(5x) + c_{2}\sin(5x)\\
\end{aligned}
\end{equation*}

\example
\begin{equation*}
\begin{aligned}
	y'' + 2y' + 6y &= 0\\
	m^{2} + 2m + 6 &= 0\\
	\frac{-b \pm \sqrt{b^{2} - 4ac}}{2a} &= \frac{-2 \pm \sqrt{2^{2} - 4(1)(6)}}{2(1)}\\
										 &= \frac{-2 \pm \sqrt{4 - 24}}{2}\\
										 &= \frac{-2 \pm \sqrt{-20}}{2}\\
										 &= \frac{-2 \pm \sqrt{4}\times\sqrt{-5}}{2}\\
										 &= \frac{-2 \pm 2\sqrt{-5}}{2}\\
										 &= -1 \pm \sqrt{-5}\\
										 &= -1 \pm \sqrt{5}i\\
	y_{1} &= e^{(-1 + \sqrt{5}i)x}\\
		  &= e^{-x}e^{i\sqrt{5}x}\\
		  &= e^{-x}\cos(\sqrt{5}x)\\
	y_{2} &= e^{(-1 - \sqrt{5}i)x}\\
		  &= e^{-x}e^{-i\sqrt{5}x}\\
		  &= e^{-x}\sin(\sqrt{5}x)\\
\end{aligned}
\end{equation*} So the general solution is \[ y = c_{1}e^{-x}\cos(\sqrt{5}x) + c_{2}e^{-x}\sin(\sqrt{5}x) \]

In general, if $m_{1} = \alpha + i\beta, m_{2} = \alpha - i\beta$ are roots of the auxiliary equation, then $\begin{array}{cc}
		y_{1} &= e^{\alpha x}\cos(\beta x)\\
		y_{2} &= e^{\alpha x}\sin(\beta x)\\
	\end{array}$ are solutions.

\example
\begin{equation*}
\begin{aligned}
	y^{(4)} - 16y &= 0\\
	m^{4} - 16 &= 0\\
	(m^{2} - 4)(m^{2} + 4) &= 0\\
	(m - 2)(m + 2)(m^{2} + 4) &= 0\\
	m &= 2: y_{1}=e^{2x}\\
	m &= -2: y_{1}=e^{-2x}\\
	m &= 2i: \cos(2x), \sin(2x)\\
\end{aligned}
\end{equation*}
%</Section-4.3>

\end{document}