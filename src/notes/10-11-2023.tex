%! Author = Len Washington III
%! Date = 10/11/2023

% Preamble
\documentclass[title={Oct 11, 2023 Notes}]{math252notes}

% Document
\begin{document}

\setcounter{chapter}{5}
\chapter{Series Solutions of Linear Equations}\label{ch:series-solutions-of-linear-equations}
\setcounter{section}{1}
%<*Section-6.2>
\section{Second Order, Linear Homogenous DE}\label{sec:second-order-linear-homogenous-de}

\begin{equation*}
\begin{aligned}
	a_{2}(x)y'' + a_{1}(x)y' + a_{0}(x)y &= 0\\
	y'' + \frac{a_{1}(x)}{a_{2}(x)}y' + \frac{a_{0}(x)}{a_{2}(x)}y &= 0\\
	y'' + P(x)y' + Q(x)y &= 0\\
\end{aligned}
\end{equation*}

$x$'s for which $a_{2}(x) \neq 0$ will be called ordinary points\label{dfn:ordinary-points}. $x$'s for which $a_{2}(x) = 0$ will be called singular points\label{dfn:singular-points}.\\

Existence of Power Series\label{thm:existence-of-power-series}
\theorem{If $x_{0}$ is an ordinary point of the DE, then there exists two, linearly independent solution $y_{1}$, $y_{2}$ which are both in the form of power series \[ \sum_{n=0}^{\infty} c_{n}(x-x_{0})^{n} \] and these series will have radius of convergence of at least the distance from $x_{0}$ to the singular point of the DE.}

\example
Consider \[ (x^{2} + 2x + 5)y'' + xy' - 6y = 0 \]
\begin{enumerate}[label=(\roman*)]
    \item What are the singular points of the DE? \answer{\begin{equation*}
    \begin{aligned}
    	x^{2} + 2x + 5 &= 0\\
    	x &= \frac{-b \pm \sqrt{b^{2} - 4ac}}{2a}\\
    	  &= \frac{-2 \pm \sqrt{2^{2} - 4(1)(5)}}{2(1)}\\
    	  &= \frac{-2 \pm \sqrt{4 - 20}}{2}\\
    	  &= \frac{-2 \pm \sqrt{-16}}{2}\\
    	  &= \frac{-2 \pm \sqrt{16}\times\sqrt{-1}}{2}\\
    	  &= \frac{-2 \pm 4i}{2}\\
    	  &= -1 \pm 2i\\
    \end{aligned}
    \end{equation*} So $-1 + 2i$ and $-1 - 2i$ are the only singular points.}
	\item Is there a power series solution centered at $x_{0}=0$? \answer{Yes, since $x_{0}=0$ is an ordinary point, you can find \[ y_{1} = \sum_{n=0}^{\infty} c_{n}x^{n} \] and \[ y_{2} = \sum_{n=0}^{\infty} d_{n}x^{n}, \] two linearly independent solutions.}
	\item What is the minimum the radius could be for these series? \answer{As stated in the \hyperref[thm:existence-of-power-series]{theorem}, the radius is at minimum the distance from $x_{0}$ to the singular point. If you have complex singular points, calculate the distance using the complex plane graph. $\sqrt{(-1-0)^{2} + (2-0)^{2}} = \sqrt{(-1)^{2} + 2^{2}} = \sqrt{1 + 4} = \sqrt{5}$.}
	\begin{itemize}
		\item How about if we want series \[ \sum_{n=0}^{\infty} c_{n}(x-3)^{2} \] \answer{$\sqrt{(-1-3)^{2} + (-2-0)^{2}} = \sqrt{(-4)^{2} + (-2)^{2}} = \sqrt{16 + 4} = \sqrt{20} = 2\sqrt{5}$}
	\end{itemize}
\end{enumerate}

\example
Use Power Series centered at 0 (\hyperref[subsubsec:maclaurin-series]{Maclaurin Series}) to solve the DE: \[ y'' - xy = 0 \]
\begin{equation*}
\begin{aligned}
	y	&= \sum_{n=0}^{\infty} c_{n}x^{n}\\
	y'	&= \sum_{n=1}^{\infty} c_{n}nx^{n-1}\\
	y''	&= \sum_{n=2}^{\infty} c_{n}n(n-1)x^{n-2}\\
\end{aligned}
\end{equation*}\begin{equation*}
\begin{aligned}
	\sum_{n=2}^{\infty} c_{n}n(n-1)x^{n-2} - x\sum_{n=0}^{\infty} c_{n}x^{n} &= \sum_{n=2}^{\infty} c_{n}n(n-1)x^{n-2} - \sum_{n=0}^{\infty} c_{n}x^{n+1}\\
				&= \sum_{k+2=2}^{\infty} c_{k+2}(k+2)(k+2-1)x^{k} - \sum_{k-1=0}^{\infty} c_{k-1}x^{k}\\
				&= \sum_{k=0}^{\infty} c_{k+2}(k+2)(k+1)x^{k} - \sum_{k=1}^{\infty} c_{k-1}x^{k}\\
				&= \sum_{k=0}^{1} c_{k+2}(k+2)(k+1)x^{k} + \sum_{k=1}^{\infty} c_{k+2}(k+2)(k+1)x^{k} - \sum_{k=1}^{\infty} c_{k-1}x^{k}\\
				&= c_{0+2}(0+2)(0+1)x^{0} + \sum_{k=1}^{\infty} \left[ c_{k+2}(k+2)(k+1)x^{k} - c_{k-1}x^{k} \right]\\
				&= c_{2}(2)(1)(1) + \sum_{k=1}^{\infty} x^{k}\left[ c_{k+2}(k+2)(k+1) - c_{k-1} \right]\\
				&= 2c_{2} + \sum_{k=1}^{\infty} x^{k}\left[ c_{k+2}(k+2)(k+1) - c_{k-1} \right]\\
				&= 2(0) + \sum_{k=1}^{\infty} x^{k}\left[ c_{k+2}(k+2)(k+1) - c_{k-1} \right]\\
				&= 0 + \sum_{k=1}^{\infty} x^{k}\left[ c_{k+2}(k+2)(k+1) - c_{k-1} \right]\\
				&= \sum_{k=1}^{\infty} x^{k}\left[ c_{k+2}(k+2)(k+1) - c_{k-1} \right] = 0
\end{aligned}
\end{equation*}\begin{equation*}
\begin{aligned}
	c_{k+2}(k+2)(k+1) - c_{k-1} &= 0\\
	c_{k+2}(k+2)(k+1) &= c_{k-1}\\
	c_{k+2} &= \frac{c_{k-1}}{(k+2)(k+1)}\\
\end{aligned}
\end{equation*}\begin{equation*}
\begin{aligned}
	c_{0} &= \mbox{arbitrary}\\
	c_{1} &= \mbox{arbitrary}\\
	c_{2} &= 0\\
	c_{3} &= \frac{c_{0}}{(1+2)(1+1)} = \frac{c_{0}}{3\times2}\\
	c_{4} &= \frac{c_{1}}{(2+2)(2+1)} = \frac{c_{1}}{4\times3}\\
	c_{5} &= \frac{c_{2}}{(3+2)(3+1)} = \frac{0}{5\times4} = 0\\
	c_{6} &= \frac{c_{3}}{(4+2)(4+1)} = \frac{c_{0}}{3\times2}\times\frac{1}{6\times5} = \frac{c_{0}}{6\times5\times3\times2}\\
	c_{7} &= \frac{c_{4}}{(5+2)(5+1)} = \frac{c_{1}}{4\times3}\times\frac{1}{7\times6} = \frac{c_{1}}{7\times6\times4\times3}\\
	c_{8} &= \frac{c_{5}}{(6+2)(6+1)} = \frac{0}{8\times7} = 0\\
\end{aligned}
\end{equation*}\begin{equation*}
\begin{aligned}
	y &= c_{0}y_{1} + c_{1}y_{2}\\
	  &= c_{0}\left( 1 + \frac{1}{3\times2}x^{3} + \frac{1}{6\times5\times3\times2}x^{6} + \dots \right) + c_{1}\left( x + \frac{1}{4\times3}x^{4} + \frac{1}{7\times6\times4\times3} + \dots \right)\\
	  &= c_{0}\left( 1 + \frac{1}{3\times2}x^{3} + \frac{4}{6\times5\times4\times3\times2}x^{6} + \dots \right) + c_{1}\left( x + \frac{2}{4\times3\times2}x^{4} + \frac{2(5)}{7\times6\times5\times4\times3\times2} + \dots \right)\\
\end{aligned}
\end{equation*}

\example
\[ (x^{2} + 1)y'' + xy' - y = 0 \]
\begin{equation*}
\begin{aligned}
	(x^{2} + 1)y'' + xy' - y = 0\\
	y'' + \frac{x}{x^{2} + 1}y' - \frac{1}{x^{2} + 1}y = 0\\
\end{aligned}
\end{equation*}
Ordinary points: \begin{equation*}
\begin{aligned}
	x^{2} + 1 &= 0\\
	x^{2} &= -1\\
	x &= \pm\sqrt{-1}\\
	x &= \pm i\\
\end{aligned}
\end{equation*}\begin{equation*}
\begin{aligned}
	y	&= \sum_{n=0}^{\infty} c_{n}x^{n}\\
	y'	&= \sum_{n=1}^{\infty} c_{n}nx^{n-1}\\
	y''	&= \sum_{n=2}^{\infty} c_{n}n(n-1)x^{n-2}\\
\end{aligned}
\end{equation*}\begin{equation*}
\begin{aligned}
	\sum_{n=2}^{\infty} c_{n}n(n-1)x^{n-2} + \frac{x}{x^{2} + 1}\sum_{n=1}^{\infty} c_{n}nx^{n-1} - \frac{1}{x^{2} + 1}\sum_{n=0}^{\infty} c_{n}x^{n} &= 0\\ % TODO: Finish example and compare with Example 6 in Section 6.2 of the textbook (pg. 248)
\end{aligned}
\end{equation*}

%</Section-6.2>

\end{document}