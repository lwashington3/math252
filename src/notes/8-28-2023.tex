%! Author = Len Washington III
%! Date = 8/28/2023

% Preamble
\documentclass[title={Aug 28{,} 2023 Notes}]{math252notes}

% Document
\begin{document}

\setcounter{chapter}{1}
\chapter{First-Order Differential Equations}
\section{Test}
%<*Section-2.2>
\section[Separable D.E.s]{Separable Differential Equations}\label{sec:separable-differential-equations}
Separable D.E.s\label{dfn:separable-de} are DE's $\frac{dy}{dx}=f(x,y)$ where $f(x,y)$ can be factored as $f(x,y)=g(x)h(y)$.

\begin{gather*}
    \frac{dy}{dx}=(1+y^{2})x^{3} \mbox{ is separable}\\
    \frac{dy}{dx}=\sin(xy) \mbox{ is \emph{not} separable}\\
    \frac{dy}{dx}=x^{3}y \mbox{ is separable}\\
\end{gather*}

\begin{equation*}
\begin{aligned}
	\frac{5}{xy}\frac{dy}{dx}&=\left( x^{2}+y \right)e^{y}\\
	\frac{dy}{dx}&=\frac{xy\left( x^{2}+y \right)e^{y}}{5}\\  % FIXME: The y should not be in the fraction
				 &=\frac{x\left( x^{2}+y \right)}{5}\times ye^{y}\\
\end{aligned}
\end{equation*}

\subsection{Method of Solution}\label{subsec:method-of-solution}
``Separate the variable'' to get $\frac{1}{h(y)}dy=g(x)d$ or $p(y)dy=g(x)dx$ where $p(y)=\frac{1}{h(y)}$.

\textbf{Integrate both sides}
\[ \int p(y)dy=\int g(x)dx \mbox{ and if possible, solve for }y\]

\example
\begin{equation*}
\begin{aligned}
	\frac{dy}{dx}&=\left( 1+y^{2} \right)x^{3}\\
	\int \frac{1}{1+y^{2}}dy&=\int x^{3}dx\\
	\tan^{-1}(y)+C_{1}&=\frac{x^{4}}{4}+C_{2}\\
	\tan^{-1}(y)&=\frac{x^{4}}{4}+C_{2}-C_{1}\\
	\tan^{-1}(y)&=\frac{x^{4}}{4}+C\\
	y&=\tan\left(\frac{x^{4}}{4}+C\right)\\
\end{aligned}
\end{equation*}

\example
Problem 12 from the textbook.
\begin{equation*}
\begin{aligned}
	\sin(3x)dx+2y\cos^{3}(3x)dy&=0\\
	\int -2ydy &= \int \frac{\sin(3x)}{\cos^{3}(x)}dx\\
			   &= \int \tan(3x)\sec^{2}(3x)dx\\
			   &= \int u\mbox{ }\frac{1}{3}du \mbox{ where }u=\tan(3x),\mbox{ }du=3\sec^{2}(3x)dx\\
	-2\int ydy &= \frac{1}{3}\int u\mbox{ }du + C\\
	-y^{2} &= \frac{u^{2}}{6}+C\\
		   &= \frac{ \tan^{2}(3x) }{6}+C\\
	\frac{ \tan^{2}(3x) }{6}+y^{2} &= -C\\
	\frac{ \tan^{2}(3x) }{6}+y^{2} &= C\\
\end{aligned}
\end{equation*}

Problem 25 from the textbook.
\[ x^{2}\frac{dy}{dx}=y-xy, y(-1)=-1 \]
\begin{equation*}
\begin{aligned}
	x^{2}\frac{dy}{dx}&=y-xy\\
	x^{2}\frac{dy}{dx}&=y(1-x)\\
	\frac{dy}{y}&=\frac{(1-x)}{x^{2}}dx\\
	\int \frac{dy}{y}&=\int \frac{(1-x)}{x^{2}}dx\\
	\int \frac{dy}{y}&=\int \frac{1}{x^{2}}dx - \int \frac{x}{x^{2}}dx\\
	\int \frac{dy}{y}&=\int \frac{1}{x^{2}}dx - \int \frac{1}{x}dx\\
	\ln|y|+C_{1} &=-\frac{1}{x}+C_{2} - \ln|x|+C_{3}\\
	\ln|y| &=-\frac{1}{x}-\ln|x|+C\\
	y &= e^{-\frac{1}{x}}\times e^{-\ln|x|}\times e^{C}\\
	y &= e^{-\frac{1}{x}}\times e^{-\ln|x|}\times e^{C}\\
	y &= e^{-\frac{1}{x}}\times \frac{1}{|x|} \times e^{C}\\
	y &= \frac{1}{|x|} e^{C-\frac{1}{x}}\\
	-1 &= \frac{1}{|-1|} e^{C-\frac{1}{-1}}\\
	-1 &= \frac{1}{1} e^{C-(-1)}\\
	-1 &= e^{C+1}\\
\end{aligned}
\end{equation*}
%</Section-2.2>

\end{document}