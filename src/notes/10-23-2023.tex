%! Author = Len Washington III
%! Date = 10/23/2023

% Preamble
\documentclass[title={Oct 23{,} 2023 Notes}]{math252notes}

% Document
\begin{document}

\setcounter{chapter}{6}
\chapter[The Laplace Transform]{Method of Laplace Transforms for Solving DE's}\label{ch:method-of-laplace-transforms-for-solving-de's}
\setcounter{section}{2}
%<*Section-7.3>
\section{Operational Rules}\label{sec:operational-rules}
\subsection{Operational Rules Part 1}\label{subsec:operational-rules-part-1}
Even with the reuses we know, a problem like \[ \laplace \left\{ e^{4t}t^{3} \right\} \] would require us to go to the definition until we learn some rules.

\begin{equation*}
\begin{aligned}
	\laplace\left\{ e^{4t}t^{3} \right\} &= \int_{0}^{\infty}e^{-st}e^{4t}t^{3}dt\\
										 &= \int_{0}^{\infty}e^{-(s-4)t}t^{3}dt\\
\end{aligned}
\end{equation*}

Compare with \[ \laplace\left\{ t^{3} \right\} = \int_{0}^{\infty} e^{-st}t^{3}dt = F(s)  \] where $f(t)=t^{3}$

\pdfbookmark[4]
{Operational Rule \#1 Equation}
{subsec:operational-rule-1}
%\begin{deeq}[H]
%	\caption{Operational Rule \#1}
	\begin{equation} \laplace \left\{ e^{at}f(t) \right\} = F(s-a)\label{eq:laplace-operational-rule-1}\end{equation} where $F(s) = \laplace \left\{ f(t) \right\}$
%\end{deeq}

\begin{equation*}
\begin{aligned}
	\laplace\left\{ e^{4t}t^{3} \right\} &= \frac{3!}{(s-4)^{4}}
\end{aligned}
\end{equation*}

\example
\begin{equation*}
\begin{aligned}
	\laplace \left\{ e^{-3t}\sin(5t) \right\} &= \laplace \left\{ \sin(5t) \right\} \bigg|_{s\rightarrow s+3}\\
											  &= \frac{5}{s^{2} + 25} \bigg|_{s\rightarrow s+3}\\
											  &= \frac{5}{(s+3)^{2} + 25} \bigg|_{s\rightarrow s+3}\\
											  &= \frac{5}{s^{2} + 6s + 9 + 25}\\
											  &= \frac{5}{s^{2} + 6s + 34}\\
\end{aligned}
\end{equation*}

\example
\begin{equation*}
\begin{aligned}
	\laplace \left\{ e^{-2t}\cos(6t) \right\} &= F(s+2)\\
											  &= \frac{s+2}{(s+2)^{2} + 36}\\
											  &= \frac{s+2}{s^{2} + 4s + 4 + 36}\\
											  &= \frac{s+2}{s^{2} + 4s + 40}\\
\end{aligned}
\end{equation*}

\example
Find \[ \laplace^{-1}\left\{ \frac{s+3}{s^{2}-8s + 97} \right\} \]
\begin{equation*}
\begin{aligned}
	\laplace^{-1}\left\{ \frac{s+3}{s^{2}-8s + 97} \right\} &= \laplace^{-1}\left\{ \frac{s+3}{s^{2} - 8s + 16 + 81} \right\}\\
															 &= \laplace^{-1}\left\{ \frac{s+3}{(s-4)^{2} + 81} \right\}\\
															 &= \laplace^{-1}\left\{ \frac{s+3+4-4}{(s-4)^{2} + 81} \right\}\\
															 &= \laplace^{-1}\left\{ \frac{s-4}{(s-4)^{2} + 81} \right\} + \laplace^{-1}\left\{ \frac{3+4}{(s-4)^{2} + 81} \right\}\\
															 &= \laplace^{-1}\left\{ \frac{s-4}{(s-4)^{2} + 81} \right\} + \laplace^{-1}\left\{ \frac{7}{(s-4)^{2} + 81} \right\}\\
															 &= \laplace^{-1}\left\{ \frac{s-4}{(s-4)^{2} + 9^{2}} \right\} + \laplace^{-1}\left\{ \frac{7}{(s-4)^{2} + 9^{2}} \right\}\\
															 &= e^{4t}\cos(9t) + \frac{7}{9}\laplace^{-1}\left\{ \frac{9}{(s-4)^{2} + 9^{2}} \right\}\\
															 &= e^{4t}\cos(9t) + \frac{7}{9}e^{4t}\sin(9t)\\
\end{aligned}
\end{equation*}

\subsection{Operational Rules Part 2}\label{subsec:operational-rules-part-2}
Involves taking the Laplace transform of a function shifted on the $t$-axis.

This can be written in terms of the Heavyside Function\label{dfn:heavyside-function} (or Unit Step function) \[ \unitstep(t) = \left\{ \begin{array}{cl}
	0 & \mbox{ if } t < 0\\
	1 & \mbox{ if } t \geq 0\\
\end{array} \right. \]

\[ f(t-2)\times\unitstep(t-2) \] is ``off when $t<2$ and on when $t \geq 2$.

\example
\begin{equation*}
\begin{aligned}
	\laplace \left\{ f(t-2)U(t-2) \right\} &= \int_{0}^{\infty} e^{-st}f(t-2)U(t-2)dt\\
										   &= \int_{0}^{2} e^{-st}f(t-2)U(t-2)dt + \int_{2}^{\infty} e^{-st}f(t-2)U(t-2)dt\\
										   &= \int_{0}^{2} e^{-st}f(t-2)(0)dt + \int_{2}^{\infty} e^{-st}f(t-2)(1)dt\\
										   &= \int_{0}^{2} 0dt + \int_{2}^{\infty} e^{-st}f(t-2)dt\\
										   &= \int_{2}^{\infty} e^{-st}f(t-2)dt\\
	(\mbox{ Let } v=t-2,\ dv=dt)\ \ \ 	   &= \int_{0}^{\infty} e^{-s(v+2)}f(v)dv\\
										   &= \int_{0}^{\infty} e^{-sv}e^{-2s}f(v)dv\\
										   &= e^{-2s}\int_{0}^{\infty} e^{-sv}f(v)dv\\
\end{aligned}
\end{equation*}
%</Section-7.3>

\end{document}