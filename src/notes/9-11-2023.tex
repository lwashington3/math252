%! Author = Len Washington III
%! Date = 9/11/2023

% Preamble
\documentclass[12pt]{report}

% Packages
\usepackage{titling}
\title{Sep 11, 2023 Notes}
\usepackage{math252notes}

% Document
\begin{document}

\setcounter{chapter}{3}
\chapter[Higher Order DEs]{Higher Order Differential Equations}\label{ch:higher-order-differential-equations}
%<*Section-4>
%\section{Linear Equations}\label{sec:linear-equations}
\noindent An $n$th order DE is linear if it had the form \[ a_{n}(x)\frac{d^{n}y}{dx^{n}} + a_{n-1}(x)\frac{d^{n-1}y}{dx^{n-1}} + a_{n-2}(x)\frac{d^{n-2}y}{dx^{n-2}} + \dots + a_{1}(x)\frac{dy}{dx}+ a_{0}y = g(x)\]

\noindent\theorem{If all the coefficient functions are continuous and $a_{n}(x)$ is not 0 on an interval $I$ and $g(x)$ is continuous, then any initial value problem \[ DE + y(x_{0}) = y_{0} \] has a unique solution on the interval $I$ if $g(x)=0$. i.e. \[ a_{n}(x)y^{(n)} + \dots + a_{0}(x)y = 0 \] then the DE is said to be homogeneous.}

\example
\[ y'' - 3y' - 4y = 0 \]
Show $y_{1}=e^{4x}$ is a solution and $y_{2}=e^{-x}$ is a solution.
\begin{equation*}
\begin{aligned}
	y_{1} &= e^{4x}\\\\
	y_{1}' &= 4e^{4x}\\\\
	y_{1}'' &= 16e^{4x}\\\\
	16e^{4x} - 3(4e^{4x}) - 4e^{4x} &= 0\\
	16e^{4x} - 12e^{4x} - 4e^{4x} &= 0\\
	e^{4x} (16 - 12 - 4) &= 0\\
	e^{4x} (0) &= 0\\
	0 &= 0
\end{aligned}
\end{equation*}

\begin{equation*}
\begin{aligned}
	y_{3} = 6y_{1} &= 6e^{4x}\\\\
	y_{3}' = 6y_{1}' &= 24e^{4x}\\\\
	y_{3}'' = 6y_{1}'' &= 96e^{4x}\\\\
	96e^{4x} - 3(24e^{4x}) - 4(6e^{4x}) &= 0\\
	96e^{4x} - 72e^{4x} - 24e^{4x} &= 0\\
	e^{4x} (96 - 72 - 24) &= 0\\
	e^{4x} (0) &= 0\\
	0 &= 0
\end{aligned}
\end{equation*}

\theorem{Superposition Principle: if $y_{1}, y_{2}, \dots, y_{m}$ are each solutions of an $n$th order Linear, homongenous DE, then $c_{1}y_{1} + c_{2}y_{2} + \dots + c_{m}y_{m}$ will also be a solution for any constants $c_{1}, c_{2}, \dots, c_{m}$.}\\

\noindent Our goal is to express the general solution in as concise a way as possible.\\

\noindent\definition{Linear combination\label{dfn:linear-combination}}{a colelction of solutions $y_{1}, y_{2}, \dots, y_{m}$ is linearly independent is if the only way $c_{1}y_{1} + c_{2}y_{2} + \dots + c_{m}y_{m}=0$ is iff (if and only if) all of the constants $c_{1}, c_{2}, \dots, c_{m}=0$. Otherwise we say $y_{1}, y_{2}, \dots, y_{m}$ are linearly dependent.}

\theorem{If the DE is an $n$th order Linear Homogeneous equation then there will exist a collection of $n$ linearly independent solutions $y_{1}, y_{2}, \dots, y_{n}$ and the general solution will be $y_{c}=c_{1}y_{1} + c_{2}y_{2} + \dots + c_{n}y_{n}$}

One way to check for linear independence is to compile the Wronskian\label{dfn:Wronskian} \[ W(y_{1},y_{2}, \dots, y_{n})= \det \left[ \begin{array}{cccc}
	y_{1} & y_{2} & \cdots & y_{n} \\
	y_{1}' & y_{2}' & \cdots & y_{n}' \\
	\vdots & \vdots & \ddots & \vdots \\
	y_{1}^{(n-1)} & y_{2}^{(n-1)} & \cdots & y_{n}^{(n-1)} \\
\end{array} \right] \]
%</Section-4>

\end{document}